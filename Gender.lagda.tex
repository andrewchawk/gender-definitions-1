\documentclass{article}

\usepackage{newunicodechar}
\usepackage{amsmath}
\usepackage{amssymb}
\usepackage{parskip}
% The coloring distracts the author.
\usepackage[bw]{agda}
\usepackage{unicode-math}
\usepackage{physics}
\usepackage{fancyref}
\usepackage[toc]{glossaries}
\usepackage[backend=bibtex]{biblatex}
\usepackage{xcolor}
\usepackage{hyperref}
\usepackage{adjustbox}
\usepackage{soul}

\addbibresource{citationNeeded.bib}

% What is a good place for this crap?
\newunicodechar{⊤}{\ensuremath{\mathnormal{\top}}}
\newunicodechar{⊥}{\ensuremath{\mathnormal{\bot}}}
\newunicodechar{ℕ}{\ensuremath{\mathnormal{\mathbb{N}}}}
\newunicodechar{₁}{\ensuremath{\mathnormal{_1}}}
\newunicodechar{₂}{\ensuremath{\mathnormal{_2}}}
\newunicodechar{≅}{\ensuremath{\mathnormal{\cong}}}
\newunicodechar{ε}{\ensuremath{\mathnormal{\epsilon}}}
\newunicodechar{τ}{\ensuremath{\mathnormal{\tau}}}
\newunicodechar{λ}{\ensuremath{\mathnormal{\lambda}}}
\newunicodechar{ℚ}{\ensuremath{\mathnormal{\mathbb{Q}}}}
\newunicodechar{ℤ}{\ensuremath{\mathnormal{\mathbb{Z}}}}
\newunicodechar{∷}{\ensuremath{\mathnormal{\Colon}}}
\newunicodechar{⊎}{\ensuremath{\mathnormal{\uplus}}}
\newunicodechar{≈}{\ensuremath{\mathnormal{\approx}}}
\newunicodechar{≉}{\ensuremath{\mathnormal{\napprox}}}
\newunicodechar{≡}{\ensuremath{\mathnormal{\equiv}}}
\newunicodechar{≢}{\ensuremath{\mathnormal{\nequiv}}}
\newunicodechar{≤}{\ensuremath{\mathnormal{\leq}}}
\newunicodechar{⊔}{\ensuremath{\mathnormal{\sqcup}}}
\newunicodechar{≟}{\ensuremath{\mathnormal{\stackrel{?}{=}}}}
\newunicodechar{∘}{\ensuremath{\mathnormal{\circ}}}
\newunicodechar{∧}{\ensuremath{\mathnormal{\land}}}
\newunicodechar{∧}{\ensuremath{\mathnormal{\land}}}
\newunicodechar{⇒}{\ensuremath{\mathnormal{\Rightarrow}}}
\newunicodechar{⟨}{\ensuremath{\mathnormal{\langle}}}
\newunicodechar{⟩}{\ensuremath{\mathnormal{\rangle}}}
\newunicodechar{∎}{\ensuremath{\mathnormal{\blacksquare}}}
\newunicodechar{∈}{\ensuremath{\mathnormal{\in}}}
\newunicodechar{∉}{\ensuremath{\mathnormal{\notin}}}
\newunicodechar{ᵇ}{\ensuremath{\mathnormal{^\AgdaFontStyle{b}}}}
\newunicodechar{∣}{\ensuremath{\mathnormal{\lvert}}}
\newunicodechar{↭}{\ensuremath{\mathnormal{\leftrightsquigarrow}}}

\newcommand{\category}[1]{\mathsf{#1}}

\title{On Gender \&c.}
\author{la zungi no'u la nakni no'u la\ .ax.}

\makeglossaries{}

\begin{document}
\maketitle{}

\begin{abstract}
This paper presents a type-theoretical definition of \st{gender} some gender-related things, e.g., ``abstract gender''.  The author hopes that such terms will be of interest to the reader instead of, God forbid,\footnote{As a deranged mathematician, the author is bound by solemn obligation to hate any and all applications of the author's work.  xo'o cau'i} actually being of some degree of use.
\end{abstract}

\section{Recommended Reading}

\subsection{The Agda Standard Library}
This document makes use of some parts of \textcite{agdaStdlib}.  A brief overview of the parts which are used is as follows:

\begin{code}
open import Data.String
  using
    ( String
    )
\end{code}

\section{A Particularly Simple Definition}

Simply, one can define a gender type \AgdaRecord{Gender1} as follows:

\begin{code}
record Gender1 : Set where
  field
    selfDescription : String
\end{code}

At the risk of sounding a bit formal, one can say that \AgdaRecord{Gender1} is such that for all possibly-gender-having entities \(x\) and terms \AgdaBound{g} of type \AgdaRecord{Gender1}, \AgdaBound{g} is the gender of \(x\) if and only if \(x\)'s description of \(x\)'s gender is \AgdaRecord{Gender1.selfDescription} \AgdaBound{g}.

% TODO: Explain flaw.

\printbibliography{}
\end{document}
