\documentclass{article}

\usepackage{newunicodechar}
\usepackage{amsmath}
\usepackage{amssymb}
\usepackage{parskip}
% The coloring distracts the author.
\usepackage[bw]{agda}
\usepackage{unicode-math}
\usepackage{physics}
\usepackage{fancyref}
\usepackage[toc]{glossaries}
\usepackage[backend=bibtex]{biblatex}
\usepackage{xcolor}
\usepackage{hyperref}
\usepackage{adjustbox}
\usepackage{soul}
\usepackage{mfirstuc}
\usepackage{amsthm}
%\usepackage{cleveref}
\usepackage{xurl}

\theoremstyle{remark}
\newtheorem{example}{Example}
\newtheorem{theorem}{Theorem}

\newcommand{\summaryLink}[2]{\hyperref[#1]{#2}}

\addbibresource{citationNeeded.bib}

% What is a good place for this crap?
\newunicodechar{⊤}{\ensuremath{\mathnormal{\top}}}
\newunicodechar{⊥}{\ensuremath{\mathnormal{\bot}}}
\newunicodechar{ℕ}{\ensuremath{\mathnormal{\mathbb{N}}}}
\newunicodechar{₁}{\ensuremath{\mathnormal{_1}}}
\newunicodechar{₂}{\ensuremath{\mathnormal{_2}}}
\newunicodechar{≅}{\ensuremath{\mathnormal{\cong}}}
\newunicodechar{ε}{\ensuremath{\mathnormal{\epsilon}}}
\newunicodechar{τ}{\ensuremath{\mathnormal{\tau}}}
\newunicodechar{λ}{\ensuremath{\mathnormal{\lambda}}}
\newunicodechar{ℚ}{\ensuremath{\mathnormal{\mathbb{Q}}}}
\newunicodechar{ℤ}{\ensuremath{\mathnormal{\mathbb{Z}}}}
\newunicodechar{∷}{\ensuremath{\mathnormal{\Colon}}}
\newunicodechar{⊎}{\ensuremath{\mathnormal{\uplus}}}
\newunicodechar{≈}{\ensuremath{\mathnormal{\approx}}}
\newunicodechar{≉}{\ensuremath{\mathnormal{\napprox}}}
\newunicodechar{≡}{\ensuremath{\mathnormal{\equiv}}}
\newunicodechar{≢}{\ensuremath{\mathnormal{\nequiv}}}
\newunicodechar{≤}{\ensuremath{\mathnormal{\leq}}}
\newunicodechar{⊔}{\ensuremath{\mathnormal{\sqcup}}}
\newunicodechar{≟}{\ensuremath{\mathnormal{\stackrel{?}{=}}}}
\newunicodechar{∘}{\ensuremath{\mathnormal{\circ}}}
\newunicodechar{∧}{\ensuremath{\mathnormal{\land}}}
\newunicodechar{∧}{\ensuremath{\mathnormal{\land}}}
\newunicodechar{⇒}{\ensuremath{\mathnormal{\Rightarrow}}}
\newunicodechar{⟨}{\ensuremath{\mathnormal{\langle}}}
\newunicodechar{⟩}{\ensuremath{\mathnormal{\rangle}}}
\newunicodechar{∎}{\ensuremath{\mathnormal{\blacksquare}}}
\newunicodechar{∈}{\ensuremath{\mathnormal{\in}}}
\newunicodechar{∉}{\ensuremath{\mathnormal{\notin}}}
\newunicodechar{ᵇ}{\ensuremath{\mathnormal{^\AgdaFontStyle{b}}}}
\newunicodechar{∣}{\ensuremath{\mathnormal{\lvert}}}
\newunicodechar{↭}{\ensuremath{\mathnormal{\leftrightsquigarrow}}}

\newcommand{\category}[1]{\mathsf{#1}}

\MFUhyphentrue{}

\setcounter{secnumdepth}{6}

\title{On Gender \&c.}
\author{la zungi no'u la nakni no'u la\ .ax.}

\makeglossaries{}

\begin{document}
\maketitle{}

\begin{abstract}
\sloppypar{}
This paper presents \st{a type-theoretical definition of gender} type-theoretical definitions of some gender-related things, e.g., abstract gender.  The author hopes that such terms will be of interest to the reader instead of, God forbid,\footnote{As a deranged mathematician, the author is bound by solemn obligation to hate any and all applications of the author's work.  xo'o cau'i} actually being of some degree of use.
\end{abstract}

\section{Goal}
In this paper, the author tries to use Agda to define a gender identity type \(G\) such that the following conditions hold:

\begin{enumerate}
  \item \(G\) can be used to represent every gender identity.
  \item For all gender identities \(g_1\) and \(g_2\), if \(g_1\) meaningfully differs from \(g_2\), then \(G\)'s representation of \(g_1\) is meaningfully different from \(G\)'s representation of \(g_2\).\label{enum:specRuleRepresentationDifference}
\end{enumerate}

\section{Miscellaneous Stuff}

\subsection{Stuff for Agda}

\paragraph{Options}
The following Agda blip indicates that this whole document is safe, i.e., does not use \AgdaKeyword{postulate}s or anything of the sort, and uses Cubical Agda.

\begin{code}
{-# OPTIONS --safe --cubical #-}
\end{code}

\paragraph{The Module Header}
This whole document constitutes an Agda module.  Specifically, the module is \AgdaModule{Gender}, per the following declaration:

\begin{code}
module Gender where
\end{code}

\paragraph{Recommended Reading}

\subparagraph{Agda Primitives}
As one may have expected, this document makes use of some Agda primitives.  Although using such primitives without explicit ``\AgdaKeyword{import}'' statements is possible, such usage demands the use of fully-qualified identifiers, and the resulting verbosity irritates the author, who, accordingly, has decided to use some ``\AgdaKeyword{open} \AgdaKeyword{import}'' statements to decrease such verbosity.  Anyway, a brief listing or such of the primitives which are used is as follows:

\begin{code}
open import Agda.Primitive.Cubical using (I)
open import Agda.Primitive using (Level ; lsuc ; SSet)
\end{code}

\subparagraph{The Agda Standard Library}
This document also makes use of some Agda modules which are defined in \textcite{agdaStdlib}.  A brief overview of the parts which are used is as follows:

\begin{code}
open import Data.String using (String)
open import Data.Maybe using (Maybe)
open import Data.List using (List)
open import Data.Rational using (ℚ ; 0ℚ ; 1ℚ)
\end{code}

\section{A Particularly Simple Definition}

\begin{code}
module Attempt1 where
\end{code}

\subsection{Summary, what with the Spoilers and Such}
\summaryLink{sec:gender1definition}{A gender type \(G\) can be defined such that for all genders \(x\), the \(G\) term which corresponds to \(x\) is\footnote{The word ``is'' is a bit of an oversimplification.} a description of \(x\), per a \gls{gEntity} whose gender is the \(G\) gender.}  However, \summaryLink{sec:gender1flaws}{this approach is flawed} in \summaryLink{sec:gender1non1gender}{being all but incompatible with \glspl{gEntity} whose gender count is not one}, not facilitating equality checking, and \summaryLink{sec:gender1vagueness}{possibly just being too vague}.

\subsection{Definition}\label{sec:gender1definition}
Simply, one can define a gender type \AgdaRecord{Gender} as follows:

\begin{code}
  record Gender : Set where
    field
      selfDescription : String
\end{code}

At the risk of sounding a bit formal, one can say that \AgdaRecord{Gender} is such that for all \glspl{gEntity} \(x\) and terms \AgdaBound{g} of type \AgdaRecord{Gender}, \AgdaBound{g} is the gender of \(x\) if and only if \(x\)'s description of \(x\)'s gender is \AgdaRecord{Gender.selfDescription} \AgdaBound{g}.  However, as indicated by the mere \emph{existence} of \fref{sec:gender1flaws}, \AgdaRecord{Gender} is flawed.

\subsection{Examples}

\begin{code}
  module Example where
\end{code}

\begin{example}\label{example:gender1exampleFemale}
  Using \AgdaRecord{Gender}, one can represent the female gender as follows:

  \begin{code}
    female : Gender
    female = record {selfDescription = "female"}
  \end{code}

  The author has no \emph{major} qualms with this approach but \hyperref[enum:gender1stringComparison]{\emph{does} think that \AgdaPostulate{String} might be a pain for determining equivalences between genders}.
\end{example}

\begin{example}
  Much like for \AgdaFunction{female}, with the assistance of \AgdaRecord{Gender}, one can represent the male gender as follows:

  \begin{code}
    male : Gender
    male = record {selfDescription = "male"}
\end{code}

\AgdaFunction{female}'s \AgdaPostulate{String} complaint also applies to \AgdaFunction{male}.
\end{example}

\begin{example}
  In addition, nonbinary genders are supported; because \AgdaField{selfDescription}s are terms of \AgdaPostulate{String}, any such self-description is appropriate.  Accordingly, one can represent, for example, the maverique gender\cite{nonbinaryWikiMaverique} as follows:

  \begin{code}
    maverique : Gender
    maverique = record {selfDescription = "maverique"}
  \end{code}

Once again, that one \AgdaPostulate{String} complaint also applies here.
\end{example}

\begin{example}
  Also, still, one can define a \AgdaRecord{Gender} term which is allegedly suitable for use with agender \glspl{gEntity} as follows:

  \begin{code}
    agender : Gender
    agender = record {selfDescription = "agender"}
  \end{code}

  However, as indicated by ``allegedly'' and the overall janky phrasing, per \fref{enum:gender1agender}, \AgdaFunction{agender} is bogus.
\end{example}

\subsection{Flaws}\label{sec:gender1flaws}
Essentially, the flaws of \AgdaRecord{Gender} are as follows:

\begin{enumerate}
  \item As indicated in \fref{sec:gender1non1gender}, \AgdaRecord{Gender} sucks for agender \glspl{gEntity}\ldots{}\label{enum:gender1agender}
  \begin{itemize}
    \item \ldots and bigender \glspl{gEntity}, trigender \glspl{gEntity}, and, generally, \(n\)-gender \glspl{gEntity}, where \(n \neq 1\).
  \end{itemize}
  \item As further explained in \fref{sec:gender1vagueness}, \AgdaPostulate{String} \emph{might} be a bit too vague.
  \item \AgdaPostulate{String} does not facilitate the equality comparison of genders.\label{enum:gender1stringComparison}
\end{enumerate}

\paragraph{\ecapitalisewords{\glspl{gEntity}} whose Gender Count Is Not One}\label{sec:gender1non1gender}
Basically, \AgdaRecord{Gender} is unsuitable for \glspl{gEntity} who are agender, bigender, trigender, or such; using \AgdaRecord{Gender} to represent gender identities only directly accommodates \glspl{gEntity} who have exactly \emph{one} gender.
\subparagraph{Agender Support}
As stated in \fref{enum:gender1agender}, \AgdaRecord{Gender} does not accommodate agender \glspl{gEntity}; such \glspl{gEntity} are likely to say that,\cite{healthline-agender}\cite{lgbtqia-wiki-agender} per the name, which essentially means ``without gender'',\cite{lgbtqia-wiki-agender} agender \glspl{gEntity} lack genders and, therefore, descriptions of such genders; accordingly, one can conclude that \AgdaRecord{Gender}, which depends upon gender self-descriptions, is unsuitable for use with agender \glspl{gEntity}.
\subparagraph{\(n\)-Gender Support}
For similar reasons, \AgdaRecord{Gender} is all but incompatible with bigender \glspl{gEntity} and such; although \AgdaRecord{Gender} \emph{does} \emph{technically} support the selection of multiple genders through comma-separated lists and such, the author finds that such second-class support is unfair and, therefore, unsuitable.  Also, like agender \glspl{gEntity}, an \(n\)-gender \gls{gEntity} \(e\), where \(n > 1\), may say that the gender identity of \(e\) is a combination of genders but \emph{is not actually a gender}, as bigender \glspl{gEntity} and such have multiple genders, not just a single gender;\cite{nonbinaryWikiBigender} in this case, the simple \AgdaRecord{Gender} approach is outright unsuitable.

\paragraph{Vagueness}\label{sec:gender1vagueness}
The author thinks that \AgdaRecord{Gender}'s \AgdaPostulate{String} approach \emph{may} be excessively vague, as this \AgdaPostulate{String} approach permits saying that the truncated output of \texttt{/dev/urandom} is a gender self-description.  However, the author acknowledges that gender self-descriptions can vary wildly and that specifically enumerating every gender is infeasible; similarly, the author is unaware of a good function \AgdaFunction{IsValidGenderSelfDescription} \AgdaSymbol{:} \AgdaPostulate{String} \AgdaSymbol{→} \AgdaPostulate{Set}.

\section{A More Agender-Friendly Type}

\begin{code}
module Attempt2 where
\end{code}

\subsection{Summary, what with the Spoilers and All}
Instead of the using the previous type, \summaryLink{sec:gender2definition}{one can say that \AgdaFunction{Maybe} \AgdaRecord{Attempt1.Gender} is the type of gender identities, with \AgdaInductiveConstructor{nothing} representing agender identities, and \AgdaInductiveConstructor{just} \AgdaBound{b} representing the gender identity which is characterized by the \AgdaBound{b} gender}.  \summaryLink{sec:gender2flaws}{But this idea is flawed, too}; this approach \summaryLink{enum:gender2nGender}{lacks support for multi-gender identities} \emph{and} \summaryLink{enum:gender2problemInheritance}{inherits all non-agender-identity-related flaws of the na\"ive use of \AgdaRecord{Attempt1.Gender}}.

\subsection{Definition}\label{sec:gender2definition}
In addition to a gender type, one can provide an ``abstract gender'' type, i.e., a type for gender identities, as opposed to a strict gender type; some \glspl{gEntity} lack genders, but the author assumes that all such \glspl{gEntity} \emph{do} have gender identities, i.e., identities which pertain to the concept of gender, although being unaware of such identity is possible.  Anyway, the structural definition of \AgdaFunction{AbstractGender}, which is the type for gender identities, is as follows:

\begin{code}
  AbstractGender : Set
  AbstractGender = Maybe Attempt1.Gender
\end{code}

Again, at the risk of sounding a bit formal, one can say that for all \AgdaRecord{Attempt1.Gender} terms \AgdaBound{g} and \glspl{gEntity} \(x\), the following statements apply:

\begin{itemize}
  \item \AgdaInductiveConstructor{just} \AgdaBound{g} is the \AgdaRecord{AbstractGender} for \(x\) if and only if the following conditions hold:
  \begin{itemize}
    \item \(x\) has a gender, and
    \item \(x\)'s gender is \AgdaBound{g}.
  \end{itemize}
  \item \AgdaInductiveConstructor{nothing} is the \AgdaRecord{AbstractGender} for \(x\) if and only if \(x\) is agender.
\end{itemize}

\subsection{Flaws}\label{sec:gender2flaws}
Some flaws of this approach are as follows:

\begin{enumerate}
  \item This approach inherits the \(n\)-gender \gls{gEntity} flaws of \AgdaRecord{Attempt1.Gender}, where \(n > 1\).\label{enum:gender2nGender}
  \item Also, this approach inherits all of the issues which are inherent in the use of \AgdaPostulate{String}; ultimately, \AgdaFunction{AbstractGender} uses \AgdaPostulate{String} for gender representation.\label{enum:gender2problemInheritance}
\end{enumerate}

\section{A More \(n\)-Gender-Friendly Approach}

\begin{code}
module Attempt3 where
\end{code}

\subsection{Summary, what with the Spoilers and All}
Alternatively \emph{still}, \summaryLink{sec:gender3definition}{one can say that for all \glspl{gEntity} \(x\), the gender identity of \(x\) is a list of all genders with which \(x\) identifies}.  This approach is suitable for agender, monogender, and polygender \glspl{gEntity}.  But \summaryLink{sec:gender2flaws}{even \emph{this} approach is flawed}; this approach \summaryLink{enum:gender2genderfluidOrPolygender}{fails to distinguish between polygender and genderfluid \glspl{gEntity}}.

\subsection{Definition}\label{sec:gender3definition}
As indicated in \fref{enum:gender2nGender}, \AgdaRecord{Attempt2.AbstractGender}, despite accommodating agender \glspl{gEntity}, fails to properly accommodate \(n\)-gender \glspl{gEntity}.  Accordingly, in the hope of better accommodating such \glspl{gEntity}, the author has structurally defined a gender identity type \AgdaFunction{AbstractGender} as follows:

\begin{code}
  AbstractGender : Set
  AbstractGender = List Attempt1.Gender
\end{code}

Semantically, one can say that the \AgdaFunction{AbstractGender} of a given \gls{gEntity} \(x\) is the list of all genders of \(x\); accordingly, the following statements hold:

\begin{enumerate}
  \item Agender \glspl{gEntity}' \AgdaFunction{AbstractGender} is the empty list, i.e., \AgdaInductiveConstructor{[]}.
  \item A given \gls{gEntity} \(x\) has exactly one gender \AgdaBound{g} if and only if the \AgdaFunction{AbstractGender} for \(x\) is \AgdaOperator{\AgdaFunction{[}} \AgdaBound{g} \AgdaOperator{\AgdaFunction{]}}, i.e., the singleton list which contains \AgdaBound{g}.
  \item A given \gls{gEntity} \(x\) has \(n\) genders, where \(n > 1\), only if the \AgdaFunction{AbstractGender} for \(x\) is a list of all genders of \(x\).\footnote{All right, this one is damn near just a repeat of the beginning of this paragraph.  Twenty lashes for the author!}
\end{enumerate}

\subsection{Flaws}\label{sec:gender3flaws}
Some flaws of this new approach are as follows:

\begin{enumerate}
  \item \hyperref[sec:gender3DistinctionFailure]{\AgdaFunction{AbstractGender} fails to explicitly distinguish between genderfluid identities and polygender identities.}\label{enum:gender2genderfluidOrPolygender}
\end{enumerate}

\paragraph{Genderfluid-Polygender Distinction Failure}\label{sec:gender3DistinctionFailure}
\AgdaFunction{AbstractGender} does not particularly well facilitate distinguishing between genderfluid identities and polygender identities.  The naive approach for representing a polygender identity with \AgdaFunction{AbstractGender} is the listing of the appropriate genders; similarly, the naive approach for representing a genderfluid identity is the listing of the appropriate genders; but, per this approach, given a polygender identity \(p\) and a genderfluid identity \(g\), if the genders of \(p\) are the genders of \(g\), then the naive \AgdaFunction{AbstractGender} representation of \(p\) is equivalent to the naive \AgdaFunction{AbstractGender} representation of \(g\).  This situation is Bad News; there exist a polygender identity \(p\) and a genderfluid identity \(f\) such that the following conditions hold:

\begin{enumerate}
  \item The genders of \(p\) are the genders of \(f\).
  \item \(p\) meaningfully differs from \(f\).
\end{enumerate}

Accordingly, one can rightfully conclude that \AgdaFunction{AbstractGender} is in violation of \fref{enum:specRuleRepresentationDifference} and, therefore, is not a good solution.

\section{A More Genderfluid-Friendly Definition}

\begin{code}
module Attempt4 where
\end{code}

\subsection{Summary, what with the Spoilers and All}
Another idea is that \summaryLink{sec:gender4definition}{gender identity is a relation on the universe contexts and the genders}.  Despite enabling representation --- and differentiation --- of agender identities, polygender identities, and genderfluid identities, \summaryLink{sec:gender4flaws}{this approach \emph{still} sucks}; \summaryLink{enum:gender4flawsClunky}{the universe context approach is clunky}, and \summaryLink{enum:gender4flawsExtent}{a simple relation does not indicate the extent of identification with a given gender}.

\subsection{Definition}\label{sec:gender4definition}
Alternatively, gender identity can be represented as a binary predicate \(P\) on the universe contexts and the genders such that for all such gender identities \(P\) and \glspl{gEntity} \(x\), \(P\) is the gender identity of \(x\) if and only if for all given universe contexts \(U\) and genders \(g\), \(P\ U\ g\) is inhabited if and only if \(g\) is a gender of \(x\) when \(x\)'s universe context is \(U\).  \AgdaFunction{AbstractGender}, which is a universe-level-polymorphic version of \(P\), is defined as follows:

\begin{code}
  AbstractGender : (u : Level) → Set (lsuc u)
  AbstractGender u = Set u → Attempt1.Gender → Set u
\end{code}

Much like the simpler \(P\), \AgdaFunction{AbstractGender} can be semantically defined as follows: For all universe levels \AgdaBound{u}, \AgdaFunction{AbstractGender} \AgdaBound{u} terms \AgdaBound{G}, and \glspl{gEntity} \(x\), \AgdaBound{G} is the gender identity of \(x\) if and only if for all universe contexts \AgdaBound{U} and genders \AgdaBound{g}, ``\AgdaBound{g} is a gender of \(x\) in context \AgdaBound{U}'' holds if and only \AgdaBound{G} \AgdaBound{U} \AgdaBound{g} is inhabited.

\subsection{Flaws}
A subset of the flaws of this approach is as follows:

\begin{enumerate}
  \item The whole ``universe context'' idea is pretty clunky.\label{enum:gender4flawsClunky}
  \item \hyperref[sec:gender4Extent]{Describing the extent of gender is unsupported.}\label{enum:gender4flawsExtent}
\end{enumerate}

\paragraph{Extent}\label{sec:gender4Extent}
Genderfluidity is such that in a given context, a genderfluid \gls{gEntity} \(x\) can identify with multiple genders at once such that the strength of \(x\)'s identification is not consistent among the genders.  Equivalently but more formally, one can say that a \gls{gEntity} \(x\), a level \AgdaBound{u}, an \AgdaFunction{AbstractGender} \AgdaBound{u} term \AgdaBound{G}, a universe context \AgdaBound{U}, and genders \AgdaBound{g1} and \AgdaBound{g2} exist such that the following conditions hold:

\begin{enumerate}
  \item \AgdaBound{G} is the gender identity of \(x\).
  \item \AgdaBound{G} \AgdaBound{U} \AgdaBound{g1} and \AgdaBound{G} \AgdaBound{U} \AgdaBound{g2} are populated.
  \item In context \AgdaBound{U}, \(x\)'s feeling of \AgdaBound{g1} is stronger than \(x\)'s feeling of \AgdaBound{g2}.
\end{enumerate}

\subparagraph{\st{Epic} Monic Fail}
Because \AgdaFunction{AbstractGender} uses proofs, not measures of extent, \AgdaFunction{AbstractGender} cannot represent such intensity variation; accordingly, the morphism from the gender identity object to the \AgdaFunction{AbstractGender} object is \emph{not} monic and, therefore, is unacceptable for this paper.

\section{A Similar --- but Interval-Based --- Approach}\label{sec:gender5}

\begin{code}
module Attempt5 where
\end{code}

\subsection{A Crude --- and Spoiler-Inclusive --- Summary}
Yet again alternatively, \summaryLink{sec:gender5definition}{gender identity can be represented as a rational-unit-interval-valued function on the universe contexts and the genders, with the rational indicating the degree of identification with the given gender in the given universe context}.  But, yet again, \summaryLink{sec:gender5flaws}{this new approach is flawed}; this approach \summaryLink{enum:gender5flawsIrrational}{does not account for irrational degrees of identification}.

\subsection{Definition}\label{sec:gender5definition}
Alternatively \emph{still} --- ooh-wee ---, one can say that gender identity is a function \(G\) from the universe contexts and the genders to the rationals which are inclusively bounded by \(1\) and \(0\) such that for all \glspl{gEntity} \(x\) and gender identities \(G\), \(G\) is the gender identity of \(x\) if and only if the extent to which \(x\) identifies with any given gender \(g\) in any given context \(U\) is \(G\ U\ g\).  Accordingly, the author has defined \AgdaFunction{AbstractGender}, which is a universe-level-abstract equivalent of this function, and a supporting interval type \AgdaRecord{Unitℚ}, as follows:

\begin{code}
  record Unitℚ : Set where
    field
      q : ℚ
      greater : 0ℚ Data.Rational.≤ q
      less : q Data.Rational.≤ 1ℚ

  AbstractGender : (u : Level) → Set (lsuc u)
  AbstractGender u = Set u → Attempt1.Gender → Unitℚ
\end{code}

For \AgdaFunction{AbstractGender}, one can \emph{so informally} say that for all levels \AgdaBound{u}, \glspl{gEntity} \(x\), and \AgdaFunction{AbstractGender} \AgdaBound{u} terms \AgdaBound{G}, \AgdaBound{G} is the gender identity of \(x\) if and only if for all relevant universe-contexts \AgdaBound{U} and genders \AgdaBound{g}, the extent to which \(x\), in \AgdaBound{U}, identifies with \AgdaBound{g} is \AgdaBound{G} \AgdaBound{U} \AgdaBound{g}, with \AgdaFunction{0ℚ} representing a complete lack of identification with a given gender and \AgdaFunction{1ℚ} representing total identification with a given gender.

\subsection{Flaws}\label{sec:gender5flaws}
Even \emph{this} approach is flawed!  Accordingly, the author has elected to list some of the flaws, which are as follows:

\begin{enumerate}
  \item \hyperref[sec:gender5irrational]{\AgdaFunction{AbstractGender} fails to accommodate irrational extents.}\label{enum:gender5flawsIrrational}
\end{enumerate}

\paragraph{Irrational Extents}\label{sec:gender5irrational}
\AgdaFunction{AbstractGender} fails to accommodate a gender identity \(P\) which is such that for all \glspl{gEntity} \(x\) of gender identity \(P\), a universe context \(U\) and a gender \(g\) exist such that the extent to which \(x\) identifies with \(g\) in \(U\) \emph{cannot} be expressed as a natural number.  Accordingly, an approximation must be used, but such approximation would violate \fref{enum:specRuleRepresentationDifference}.  Accordingly, this approach is \emph{also} unsuitable.

\section{The Continuous Interval Type}\label{sec:gender6}

\begin{code}
module Attempt6 where
\end{code}

\subsection{Summary and Spoilers}
\summaryLink{sec:gender6definition}{This \AgdaFunction{AbstractGender} is pretty much just a real-interval-valued version of the \AgdaFunction{AbstractGender} of \fref{sec:gender5}}.

\subsection{Definition}\label{sec:gender6definition}
Much like whoever thought that \AgdaFunction{Attempt5.AbstractGender} is a good idea, one can say that gender identity can be represented as a real-unit-interval-valued binary function \(G\) such that for all \glspl{gEntity} \(x\), \(G\) is the gender identity of \(x\) if and only if for all genders \(g\) and universe contexts \(U\), the extent of \(x\)'s identification with \(g\) in \(U\) is \(G\ U\ g\).  One can define a corresponding Agda function \AgdaFunction{AbstractGender} as follows:

\begin{code}
  AbstractGender : (u : Level) → SSet (lsuc u)
  AbstractGender u = Set u → Attempt1.Gender → I
\end{code}

Once again, the Agda version is essentially just a universe-polymorphic equivalent of the non-Agda function.  More formally, one, e.g., the author, who is sick of writing this boilerplate-like stuff, can say that for all \glspl{gEntity} \(x\), levels \AgdaBound{u}, and \AgdaFunction{AbstractGender} \AgdaBound{u} terms \AgdaBound{G}, \AgdaBound{G} is the gender identity of \(x\) if and only if for all appropriate \AgdaBound{U} and \AgdaBound{g}, the extent to which \(x\) identifies with \AgdaBound{g} in \AgdaBound{U} is \AgdaBound{G} \AgdaBound{U} \AgdaBound{g}, where \AgdaInductiveConstructor{Agda.Primitive.Cubical.i0} and \AgdaInductiveConstructor{Agda.Primitive.Cubical.i1} indicate complete non-identification and total identification with \AgdaBound{g}, respectively, and other \AgdaDatatype{I} values are assigned linearly.

\subsection{Flaws}\label{sec:gender6flaws}
Some flaws of this approach are as follows:

\begin{enumerate}
  \item \AgdaRecord{Gender} only permits the use of \emph{one} gender self-description.\label{enum:gender6flawsMultiDesc}
\end{enumerate}

\section{An Alternative Type for Gender}

\begin{code}
module Attempt7 where
\end{code}

\subsection{Summary}
This \AgdaFunction{AbstractGender} is quite similar to \hyperref[sec:gender6]{\AgdaFunction{Attempt6.AbstractGender}}, with the exception of the underlying type for gender; in this case, genders are essentially \AgdaFunction{Set}-level functions \(f\); this approach permits the use of multiple valid self-descriptions of a single gender.  But the identity type is still a binary function on the universe contexts and the genders.

\subsection{Definitions}
\paragraph{A New Type for Gender}
In contrast with \AgdaFunction{Attempt1.Gender}, one can say that gender can be represented as a unary relation \(f\) on \AgdaPostulate{String} such that for all \(s\), \(f\ s\) is populated if and only if the use of \(s\) as a gender self-description for \(f\) is valid.  \AgdaRecord{Gender}, the Agda equivalent of \(f\), is defined as follows:

\begin{code}
  record Gender : Set1 where
    field
      selfDescriptions : String → Set
\end{code}

\paragraph{The Gender Identity Type}
Pretty much exactly like in \fref{sec:gender6definition} --- actually, the definition is so similar that the author will not even bother with writing another semantic definition ---, one can define a gender identity type \AgdaFunction{AbstractGender} as follows:

\begin{code}
  AbstractGender : (u : Level) → SSet (lsuc u)
  AbstractGender u = Set u → Gender → I
\end{code}

\subsection{Advantages}\label{sec:gender7advantages}
Some advantages of this approach are as follows:

\begin{enumerate}
  \item This approach solves the ``one description'' problem of \fref{enum:gender6flawsMultiDesc}.
\end{enumerate}

\subsection{Flaws}\label{sec:gender7flaws}
Some flaws of this approach are as follows:

\begin{enumerate}
  \item \summaryLink{sec:gender7flawExtent}{\AgdaRecord{Gender} does not indicate the \emph{extent} of identification with a given gender description.}\label{enum:gender7flawExtent}
\end{enumerate}

\paragraph{Extent}\label{sec:gender7flawExtent}
\begin{theorem}
  This approach is bogus.
\end{theorem}

\begin{proof}
  This approach is bogus if there exist two distinct genders \(g_1\) and \(g_2\) such that \AgdaRecord{Gender} cannot differentiate \(g_1\) and \(g_2\).  Such distinct genders exist if genders \(g_1\) and \(g_2\) and a gender self-description \(s\) exist such that the following statement holds: \(g_1\) \glspl{gEntity} and \(g_2\) \glspl{gEntity} \emph{both} identify with \(s\), but the \emph{extent} of \(g_1\)'s identification with \(s\) is greater than the extent of \(g_2\)'s identification with \(s\).  There \emph{do} exist such genders, and there \emph{does} exist such a gender self-description.  Therefore, this approach is bogus.
\end{proof}
\section{The Interval Type for Gender}

\begin{code}
module Attempt8 where
\end{code}

\subsection{Definitions}

\paragraph{Yet Another Type for Gender}
In contrast with \AgdaFunction{Attempt7.Gender}, which does not indicate the extent of identification with a given gender description, one can say that for all genders \(g\), \(g\) is a unary function from the gender self-descriptions, i.e., \AgdaPostulate{String} terms, to the unit interval, with \(0\) indicating complete non-identification with a given description and \(1\) indicating total\footnote{Total identification with multiple distinct gender self-descriptions is possible.}  identification with a given description.  As Dear Reader may have expected, the other interval values are assigned linearly.  With this idea in mind, one can define an Agda type \AgdaRecord{Gender}, in which \AgdaField{Gender.selfDescriptions} serves as the container for these fancy ``\(g\)'' functions, as follows:

\begin{code}
  record Gender : SSet where
    field
      selfDescriptions : String → I
\end{code}

\paragraph{The Gender Identity Type}
Again, pretty much exactly like in \fref{sec:gender6definition} --- actually, the definition is so similar that the author will, \emph{again}, not even bother with writing another semantic definition ---, one can define a gender identity type \AgdaFunction{AbstractGender} as follows:

\begin{code}
  AbstractGender : (u : Level) → SSet (lsuc u)
  AbstractGender u = Set u → Gender → I
\end{code}

\subsection{Advantages}
Some advantages of this approach are as follows:

\begin{enumerate}
  \item This approach \emph{does} solve that identification extent problem which was described in \fref{enum:gender7flawExtent}.
\end{enumerate}

\printbibliography{}

\newglossaryentry{gEntity}
  { name = {G-entity}
  , plural = {G-entities}
  , description = Being a G-entity is materially equivalent to being something which could have a gender
  , see =[cf.]{gWagon}
  }

\newglossaryentry{gWagon}
  { name = {G-Wagon}
  , plural = {G-Wagons}
  , description = \glspl{gWagon} are MERCEDES-BENZ G-Class sports utility vehicles
  , see =[cf.]{gEntity}
  }

\glsadd{gWagon}

\printglossary{}
\end{document}
