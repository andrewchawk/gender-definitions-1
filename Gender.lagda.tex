\documentclass{article}

\usepackage{newunicodechar}
\usepackage{amsmath}
\usepackage{amssymb}
\usepackage{parskip}
% The coloring distracts the author.
\usepackage[bw]{agda}
\usepackage{unicode-math}
\usepackage{physics}
\usepackage{fancyref}
\usepackage[toc]{glossaries}
\usepackage[backend=bibtex]{biblatex}
\usepackage{xcolor}
\usepackage{hyperref}
\usepackage{adjustbox}
\usepackage{soul}
\usepackage{mfirstuc}

\addbibresource{citationNeeded.bib}

% What is a good place for this crap?
\newunicodechar{⊤}{\ensuremath{\mathnormal{\top}}}
\newunicodechar{⊥}{\ensuremath{\mathnormal{\bot}}}
\newunicodechar{ℕ}{\ensuremath{\mathnormal{\mathbb{N}}}}
\newunicodechar{₁}{\ensuremath{\mathnormal{_1}}}
\newunicodechar{₂}{\ensuremath{\mathnormal{_2}}}
\newunicodechar{≅}{\ensuremath{\mathnormal{\cong}}}
\newunicodechar{ε}{\ensuremath{\mathnormal{\epsilon}}}
\newunicodechar{τ}{\ensuremath{\mathnormal{\tau}}}
\newunicodechar{λ}{\ensuremath{\mathnormal{\lambda}}}
\newunicodechar{ℚ}{\ensuremath{\mathnormal{\mathbb{Q}}}}
\newunicodechar{ℤ}{\ensuremath{\mathnormal{\mathbb{Z}}}}
\newunicodechar{∷}{\ensuremath{\mathnormal{\Colon}}}
\newunicodechar{⊎}{\ensuremath{\mathnormal{\uplus}}}
\newunicodechar{≈}{\ensuremath{\mathnormal{\approx}}}
\newunicodechar{≉}{\ensuremath{\mathnormal{\napprox}}}
\newunicodechar{≡}{\ensuremath{\mathnormal{\equiv}}}
\newunicodechar{≢}{\ensuremath{\mathnormal{\nequiv}}}
\newunicodechar{≤}{\ensuremath{\mathnormal{\leq}}}
\newunicodechar{⊔}{\ensuremath{\mathnormal{\sqcup}}}
\newunicodechar{≟}{\ensuremath{\mathnormal{\stackrel{?}{=}}}}
\newunicodechar{∘}{\ensuremath{\mathnormal{\circ}}}
\newunicodechar{∧}{\ensuremath{\mathnormal{\land}}}
\newunicodechar{∧}{\ensuremath{\mathnormal{\land}}}
\newunicodechar{⇒}{\ensuremath{\mathnormal{\Rightarrow}}}
\newunicodechar{⟨}{\ensuremath{\mathnormal{\langle}}}
\newunicodechar{⟩}{\ensuremath{\mathnormal{\rangle}}}
\newunicodechar{∎}{\ensuremath{\mathnormal{\blacksquare}}}
\newunicodechar{∈}{\ensuremath{\mathnormal{\in}}}
\newunicodechar{∉}{\ensuremath{\mathnormal{\notin}}}
\newunicodechar{ᵇ}{\ensuremath{\mathnormal{^\AgdaFontStyle{b}}}}
\newunicodechar{∣}{\ensuremath{\mathnormal{\lvert}}}
\newunicodechar{↭}{\ensuremath{\mathnormal{\leftrightsquigarrow}}}

\newcommand{\category}[1]{\mathsf{#1}}

\MFUhyphentrue{}

\title{On Gender \&c.}
\author{la zungi no'u la nakni no'u la\ .ax.}

\makeglossaries{}

\begin{document}
\maketitle{}

\begin{abstract}
\sloppypar{}
This paper presents \st{a type-theoretical definition of gender} type-theoretical definitions of some gender-related things, e.g., abstract gender.  The author hopes that such terms will be of interest to the reader instead of, God forbid,\footnote{As a deranged mathematician, the author is bound by solemn obligation to hate any and all applications of the author's work.  xo'o cau'i} actually being of some degree of use.
\end{abstract}

\section{Recommended Reading}

\subsection{The Agda Standard Library}
This document makes use of some parts of \textcite{agdaStdlib}.  A brief overview of the parts which are used is as follows:

\begin{code}
open import Data.String
  using
    ( String
    )
\end{code}

\section{A Particularly Simple Definition}
Simply, one can define a gender type \AgdaRecord{Gender1} as follows:

\begin{code}
record Gender1 : Set where
  field
    selfDescription : String
\end{code}

At the risk of sounding a bit formal, one can say that \AgdaRecord{Gender1} is such that for all \glspl{gEntity} \(x\) and terms \AgdaBound{g} of type \AgdaRecord{Gender1}, \AgdaBound{g} is the gender of \(x\) if and only if \(x\)'s description of \(x\)'s gender is \AgdaRecord{Gender1.selfDescription} \AgdaBound{g}.  However, as indicated by the mere \emph{existence} of \fref{sec:gender1flaws}, \AgdaRecord{Gender1} is flawed.

\subsection{Flaws}\label{sec:gender1flaws}
Essentially, the flaws of \AgdaRecord{Gender1} are as follows:

\begin{enumerate}
  \item \hyperref[sec:gender1non1gender]{\AgdaRecord{Gender1} sucks for agender \glspl{gEntity}\ldots{}\label{enum:gender1agender}}
  \begin{itemize}
    \item \hyperref[sec:gender1non1gender]{\ldots and bigender \glspl{gEntity}, trigender \glspl{gEntity}, and, generally, \(n\)-gender \glspl{gEntity}, where \(n \neq 1\).}
  \end{itemize}
  \item \AgdaRecord{String} \emph{might} be a bit too vague.
  \item \AgdaRecord{String} does not facilitate the equality comparison of genders.
\end{enumerate}

\paragraph{\ecapitalisewords{\glspl{gEntity}} whose Gender Count Is Not One}\label{sec:gender1non1gender}
Basically, \AgdaRecord{Gender1} is unsuitable for \glspl{gEntity} who are agender, bigender, trigender, or such; using \AgdaRecord{Gender1} to represent gender identities only directly accommodates \glspl{gEntity} who have exactly \emph{one} gender.
\subparagraph{Agender Support}
As stated in \fref{enum:gender1agender}, \AgdaRecord{Gender1} does not accommodate agender \glspl{gEntity}; such \glspl{gEntity} are likely to say that,\cite{healthline-agender}\cite{lgbtqia-wiki-agender} per the name, which essentially means ``without gender'',\cite{lgbtqia-wiki-agender} agender \glspl{gEntity} lack genders and, therefore, descriptions of such genders; accordingly, one can conclude that \AgdaRecord{Gender1}, which depends upon gender self-descriptions, is unsuitable for use with agender \glspl{gEntity}.
\subparagraph{\(n\)-Gender Support}
For similar reasons, \AgdaRecord{Gender1} is all but incompatible with bigender \glspl{gEntity} and such; although \AgdaRecord{Gender1} \emph{does} \emph{technically} support the selection of multiple genders through comma-separated lists and such, the author finds that such second-class support is unfair and, therefore, unsuitable.  Also, like agender \glspl{gEntity}, an \(n\)-gender \gls{gEntity} \(e\), where \(n > 1\), may say that the gender identity of \(e\) is a combination of genders but \emph{is not actually a gender};\cite{citationNeeded} in this case, the simple \AgdaRecord{Gender1} approach is outright unsuitable.

\printbibliography{}

\newglossaryentry{gEntity}
  { name = {G-entity}
  , plural = {G-entities}
  , description = Being a G-entity is materially equivalent to being something which could have a gender
  }

\printglossary{}
\end{document}
